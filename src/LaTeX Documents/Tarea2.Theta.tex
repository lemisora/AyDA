\documentclass[11pt,letterpaper,fleqn]{article}
\usepackage[T1]{fontenc}
\usepackage[left=1in, right=1in, top=1in, bottom=1in]{geometry}
\usepackage[spanish]{babel}
\usepackage{amsmath}
\usepackage{amssymb}
\usepackage{amsthm}
\usepackage{xcolor}
\title{Caso Theta - Complejidad de algoritmos}
\author{Leonardo Michel Domingo Sánchez}
\begin{document}
\maketitle
Demostrar que $ \frac{1}{2}n^{2} - 3n = \Theta(n^{2}) $ \smallbreak
De acuerdo a la definición formal que es la siguiente:
\begin{equation}
\begin{aligned}
	\Theta(g(n)) \Leftrightarrow \{ T(n) \ | \ \exists c_{1},c_{2} > 0 \ y \ n_{0} \ \forall \ | n \geq n_{0} \ ; \ c_{1} g(n) \geq T(n) \geq c_{2}g(n)\}
\end{aligned}
\end{equation}

Para este problema se tiene: 
\begin{itemize}
    \item $T(n) = \frac{1}{2} n^{2} - 3n$
    \item $ g(n) = n^{2} $
\end{itemize}

\end{document}