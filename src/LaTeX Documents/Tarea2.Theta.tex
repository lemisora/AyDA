\documentclass[11pt,letterpaper,fleqn]{article}
\usepackage[T1]{fontenc}
\usepackage[left=1in, right=1in, top=1in, bottom=1in]{geometry}
\usepackage[spanish]{babel}
\usepackage{amsmath}
\usepackage{amssymb}
\usepackage{amsthm}
\usepackage{xcolor}
\title{Caso $\Theta$ - Complejidad de algoritmos}
\author{Leonardo Michel Domingo Sánchez}
\begin{document}
\maketitle
Demostrar que $\frac{1}{2}n^{2} - 3n = \Theta(n^{2}) $ \smallbreak
De acuerdo a la definición formal que es la siguiente:
\begin{equation*}
\begin{aligned}
	\Theta(g(n)) \Leftrightarrow \{ T(n) \ | \ \exists c_{1},c_{2} > 0 \ y \ n_{0} \ \forall \ | n \geq n_{0} \ ; \ c_{1} g(n) \geq T(n) \geq c_{2}g(n)\}
\end{aligned}
\end{equation*}

Para este problema se tiene: 
\begin{itemize}
    \item $T(n) = \frac{1}{2} n^{2} - 3n$
    \item $ g(n) = n^{2} $
\end{itemize}

Aplicando lo que dice la definición formal se tiene la siguiente inecuación:
\begin{equation*}
\begin{aligned}
    c_{1} \leq \frac{1}{2} - \frac{3}{n} \leq c_{2}
\end{aligned}
\end{equation*}
Resolviendo la inecuación:
\begin{equation*}
    c_{1} \leq \frac{n-6}{2n} \leq c_{2} 
\end{equation*}
De la definición se tiene que $C_{1}, C_{2} > 0 $
\begin{equation*}
\begin{aligned}
    0 < \frac{n-6}{2n} \ \Rightarrow \ 0 < n-6 \ \Rightarrow \ 6 < n \\
    \therefore n \geq 7& \ \text{y} \ n_{0} = 6
\end{aligned}
\end{equation*}

Encontrar las constantes
Para $c_{1}$
\begin{equation*}
\begin{aligned}
    &c_{1} \leq \frac{1}{2} - \frac{3}{7} \\
    &c_{1} \leq \frac{7-6}{14} = \frac{1}{14} \\
    &c_{1} \leq \frac{1}{14}
\end{aligned}
\end{equation*}

Para $c_{2}$
Como $n$ puede tomar cualquier valor natural entonces:
\begin{equation*}
\begin{aligned}
    &\lim\limits_{n \rightarrow \infty} \Bigg(\frac{1}{2} - \frac{3}{n}\Bigg) = \frac{1}{2} \\
    &\therefore \ c_{2} = \frac{1}{2}
\end{aligned}
\end{equation*}

Como fue posible encontrar $n_{0} = 6$ y $\ c_{1}=\frac{1}{14}, \ c_{2} = \frac{1}{2}$
$$\therefore \frac{1}{2}n^{2}-3n = \Theta(n^{2})$$
\end{document}