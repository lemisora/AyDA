\documentclass[letterpaper,11pt, fleqn]{article}
\usepackage[top=2.5cm, bottom=2.54cm, left=2.54cm, right=2.54cm]{geometry}\usepackage[T1]{fontenc}
\usepackage[utf8]{inputenc}
\usepackage[spanish]{babel}
\usepackage{lmodern}
\usepackage{amsmath}
\usepackage{amsfonts}
\usepackage{amssymb}
\usepackage{amsthm}
\usepackage{graphicx}
\usepackage{color}
\usepackage{xcolor}
\usepackage{url}
\usepackage{textcomp}
\usepackage{parskip}

\title{Sucesión de Fibonacci}
\author{Leonardo Michel Domingo Sánchez}
\date{\today}

\begin{document}

\maketitle
\tableofcontents

\begin{abstract}
	En este documento se resuelve la recurrencia de la sucesión de Fibonacci, además de que se comprueba que el resultado obtenido es correcto y equivalente con la recurrencia, comparándolos en resultados además de comprobar dicha equivalencia mediante el método de inducción
\end{abstract}

\section{Resolución de la recurrencia}
Se tiene la siguiente recurrencia con sus respectivas condiciones iniciales

\begin{equation}
\begin{cases}
\begin{aligned}
	f_{n} &= f_{n-1} + f_{n-2} \\
	f_{0} &= 0, \ f_{1} = 1
\end{aligned}
\end{cases}
\end{equation}

Para resolverla haremos cambio de variables, de la siguiente forma
$t^{n} = f_{n}$

Sustituyendo en la ecuación (1) queda como:
\begin{equation*}
	\begin{aligned}
		&t^{n} = t^{n-1} + t^{n-2} \\
		&t^{n} - t^{n-1} - t^{n-2} = 0
	\end{aligned}
\end{equation*}

Dividiendo la ecuación entre $t^{n-2}$ se tiene:
\begin{equation*}
	\frac{t^{n}}{t^{n-2}} - \frac{t^{n-1}}{t^{n-2}} - \frac{t^{n-2}}{t^{n-2}} = 0
\end{equation*}

\begin{equation}
	t^{2} - t - 1 = 0
\end{equation}

Resolviendo para $t$ se usará la fórmula general para ecuaciones cuadráticas que es la siguiente \[t = \frac{-b \pm \sqrt{b^2 - 4ac}}{2a}\] \newpage

% Segunda página
Para obtener $t$ con la fórmula se tiene que $a = 1; \ b = -1; \ c = -1$
\begin{equation*}
\begin{aligned}
	t = \frac{1 \pm \sqrt{(-1)^2 - 4(1)(-1)}}{2(1)} \Rightarrow \frac{1 \pm \sqrt{1 + 4}}{2} \Rightarrow \frac{1 \pm \sqrt{5}}{2} \\
	t_{1} = \frac{1 + \sqrt{5}}{2} \ ; \ t_{2} = \frac{1 - \sqrt{5}}{2}
\end{aligned}	
\end{equation*}

Con esto se puede demostrar que la solución es de la forma $ U_{n} = b(\frac{1 + \sqrt{5}}{2})^{n} + d(\frac{1 - \sqrt{5}}{2})^{n} $ \smallbreak
Para encontrar los coeficientes consideremos las condiciones iniciales de forma que:
\begin{equation*}
\begin{aligned}
	0 = U_{0} = b(\frac{1 + \sqrt{5}}{2})^{0} + d(\frac{1 - \sqrt{5}}{2})^{0} \Rightarrow b(1) + d(1) \Rightarrow b + d \\
	1 = U_{1} = b(\frac{1 + \sqrt{5}}{2})^{1} + d(\frac{1 - \sqrt{5}}{2})^{1} \Rightarrow b(\frac{1 + \sqrt{5}}{2}) + d(\frac{1 - \sqrt{5}}{2})\\
\end{aligned}
\end{equation*}

Se obtiene el siguiente sistema de ecuaciones
\begin{equation*}
	\begin{cases}
		\begin{aligned}
			U_{0} &= b + d = 0\\
			U_{1} &= b(\frac{1 + \sqrt{5}}{2}) + d(\frac{1 - \sqrt{5}}{2}) = 1\\
		\end{aligned}
	\end{cases}
\end{equation*}

Resolviendo el sistema de ecuaciones mediante el método de suma y resta
\begin{equation*}
	\begin{aligned}
		-(\frac{1 + \sqrt{5}}{2})(b+d) = -(\frac{1 + \sqrt{5}}{2})(0) \Longrightarrow -b(\frac{1 + \sqrt{5}}{2}) - d(\frac{1 + \sqrt{5}}{2}) = 0 \\
		b(\frac{1 + \sqrt{5}}{2}) + d(\frac{1 - \sqrt{5}}{2}) = 1
	\end{aligned}
\end{equation*}

Cancelando términos semejantes nos quedamos únicamente con lo siguiente:
\begin{equation*}
\begin{aligned}
	d(\frac{1 - \sqrt{5}}{2}) = 1\\
	d(\frac{1 + \sqrt{5}}{2}) = 0
\end{aligned}
\end{equation*}

Con lo que se tiene buscamos obtener el valor de $d$
\begin{equation*}
\begin{aligned}
	d(\frac{1 - \sqrt{5} -1 - \sqrt{5}}{2}) &= 1 \ \Rightarrow \ d(\frac{-2\sqrt{5}}{2}) = 1 \\
	d(-\sqrt{5}) &= 1 \\ d = \frac{1}{-\sqrt{5}}
\end{aligned}
\end{equation*}
\newpage

% Tercera página
Obtener el valor de $b$ al sustituir el valor de $d$ en la siguiente ecuación del sistema de ecuaciones 
\begin{equation*}
\begin{aligned}
	b + d =& 0 \\
	b + (\frac{1}{-\sqrt{5}}) &= 0 \\
	 b = \frac{1}{\sqrt{5}}
\end{aligned}
\end{equation*}

Tras resolver el sistema de ecuaciones se tiene que
\begin{equation*}
	\begin{aligned}
		b = \frac{1}{\sqrt{5}} \ ; \ d = \frac{1}{-\sqrt{5}}
	\end{aligned}
\end{equation*}

Convirtiendo a la ecuación inicial de la recurrencia $f_{n}$ a lo obtenido con la ecuación $U_{n}$ \smallbreak
Se obtiene lo siguiente:
\begin{equation*}
\begin{aligned}
	f_{n} =& \frac{1}{\sqrt{5}} (\frac{1 + \sqrt{5}}{2})^{n} - \frac{1}{\sqrt{5}} (\frac{1 - \sqrt{5}}{2})^{n} \\
	=& \frac{1}{\sqrt{5}} ((\frac{1 + \sqrt{5}}{2})^{n} - (\frac{1 - \sqrt{5}}{2})^{n})
\end{aligned}
\end{equation*}

\begin{equation*}
\begin{aligned}
	\therefore \ f_{n} = \frac{1}{\sqrt{5}} ((\frac{1 + \sqrt{5}}{2})^{n} - (\frac{1 - \sqrt{5}}{2})^{n})
\end{aligned}
\end{equation*}

Comprobando la equivalencia de la ecuación obtenida con la ecuación inicial recurrente comparándolos para $f_{6}$ \smallbreak
Con recurrencia:
\begin{equation*}
\begin{aligned}
	f_{2} = 1 + 0 = 1 \\
	f_{3} = 1 + 1 = 2 \\
	f_{4} = 2 + 1 = 3 \\
	f_{5} = 3 + 2 = 5 \\
	f_{6} = 5 + 3 = 8 \\
\end{aligned}
\end{equation*}

Sin recurrencia:
\begin{equation*}
\begin{aligned}
	f_{6} &= \frac{1}{\sqrt{5}} ((\frac{1 + \sqrt{5}}{2})^{6} - (\frac{1 - \sqrt{5}}{2})^{6}) \\
	f_{6} &= 8
\end{aligned}
\end{equation*}
\newpage

%Cuarta página
\section{Encontrar orden de complejidad del algoritmo}
De acuerdo a la ecuación obtenida de $f_{n}$ encontraremos el orden de complejidad de la ecuación:
\begin{equation*}
\begin{aligned}
	O(g(n)) =& O(\frac{1}{\sqrt{5}} ((\frac{1 + \sqrt{5}}{2})^{n} - (\frac{1 - \sqrt{5}}{2})^{n})) \ \Rightarrow \ O(\frac{1 + \sqrt{5}}{2})^{n}) \\
	\therefore O(g(n)) =& O((\frac{1 + \sqrt{5}}{2})^{n})
\end{aligned}
\end{equation*}

\section{Demostración por el método de inducción}
Hipótesis : Se cumple que: $ f_{n} = \frac{1}{\sqrt{5}} ((\frac{1 + \sqrt{5}}{2})^{n} - (\frac{1 - \sqrt{5}}{2})^{n}) $ \smallbreak
Caso base : Demostrar para $ n = 0$
\begin{equation*}
\begin{aligned}
	f_{0} &= \frac{1}{\sqrt{5}} ((\frac{1 + \sqrt{5}}{2})^{0} - (\frac{1 - \sqrt{5}}{2})^{0}) \\
	&= \frac{1}{\sqrt{5}} (1 - 1) = 0 \\
	\therefore \ f_{0} &= 0
\end{aligned}
\end{equation*}

Caso inductivo : Supongamos que se cumple
\begin{equation}
	f_{n} = \frac{1}{\sqrt{5}} \Biggl(\Biggl(\frac{1 + \sqrt{5}}{2}\Biggr)^{n} - \Biggl(\frac{1 - \sqrt{5}}{2}\Biggr)^{n}\Biggr)
\end{equation}

\qquad\qquad Por demostrar que se cumple: $f_{n+1} = \frac{1}{\sqrt{5}} ((\frac{1 + \sqrt{5}}{2})^{n+1} - (\frac{1 - \sqrt{5}}{2})^{n+1})$ \smallbreak
\qquad\qquad Si 
\begin{equation*}
\begin{aligned}
	f_{n} = f_{n-1} ((\frac{1+\sqrt{5}}{2}) - (\frac{1-\sqrt{5}}{2}))
\end{aligned}
\end{equation*}
\qquad \qquad entonces
\begin{equation}
\begin{aligned}
	f_{n+1} &= f_{n}((\frac{1+\sqrt{5}}{2}) - (\frac{1-\sqrt{5}}{2}))
\end{aligned}
\end{equation}

\qquad \qquad Sustituyendo la ecuación (3) en la ecuación (4)
\begin{equation*}
\begin{aligned}
	f_{n+1} &= (\frac{1}{\sqrt{5}} ((\frac{1 + \sqrt{5}}{2})^{n} - (\frac{1 - \sqrt{5}}{2})^{n}))((\frac{1+\sqrt{5}}{2}) - (\frac{1-\sqrt{5}}{2}))\\
	f_{n+1} &= (\frac{1}{\sqrt{5}}) ((\frac{1 + \sqrt{5}}{2})^{n+1}) - (\frac{1 - \sqrt{5}}{2})^{n+1}))
\end{aligned}
\end{equation*}
\qquad Como sí se cumple queda entonces demostrado que la solución de la recurrencia que se ha obtenido sí es correcta.
\end{document}