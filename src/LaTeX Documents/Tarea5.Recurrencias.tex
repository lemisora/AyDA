\documentclass[11pt,letterpaper,fleqn]{article}
\usepackage[T1]{fontenc}
\usepackage[left=1in, right=1in, top=1in, bottom=1in]{geometry}
\usepackage[spanish]{babel}
\usepackage{amsmath}
\usepackage{amssymb}
\usepackage{amsthm}
\usepackage{xcolor}
\title{Recurrencias}
\author{Leonardo Michel Domingo Sánchez}

\begin{document}
\maketitle
\tableofcontents
\break

\section{Recurrencias}
Resolver y encontrar el orden de las siguientes recurrencias por sustitución

\begin{equation}
	T(n) =
	\begin{cases}
		\begin{aligned}
			&3 \ si \ n = 1\\
			&n + 2T(\frac{n}{2}) \ si \ n > 1
		\end{aligned}
	\end{cases}
\end{equation}

\begin{equation}
	T(n) =
	\begin{cases}
		\begin{aligned}
			&5 \ si \ n = 1\\
			&3 + 4T(n-1) \ si \ n > 1
		\end{aligned}
	\end{cases}
\end{equation}

\begin{equation}
	T(n) =
	\begin{cases}
		\begin{aligned}
			&d \ si \ n = 1\\
			&3T(\frac{n}{2}) \ si \ n > 1
		\end{aligned}
	\end{cases}
\end{equation}

\begin{equation}
	T(n) =
	\begin{cases}
		\begin{aligned}
			&c \ si \ n = 1\\
			&T(n-1)+n \ si \ n > 1
		\end{aligned}
	\end{cases}
\end{equation}

\begin{equation}
	T(n) =
	\begin{cases}
		\begin{aligned}
			&c \ si \ n = 1\\
			&4T(n-1)+2n \ si \ n > 1
		\end{aligned}
	\end{cases}
\end{equation}

\subsection{Resolviendo la primera recurrencia}
\begin{equation*}
	T(n) =
	\begin{cases}
		\begin{aligned}
			&3 \ si \ n = 1\\
			&n + 2T(\frac{n}{2}) \ si \ n > 1
		\end{aligned}
	\end{cases}
\end{equation*}

\smallbreak Verificar valores de $n$ para los que la recurrencia tiene sentido
\begin{itemize}
	\item Para $n=1$ tiene sentido la recurrencia
	\item Para $n=2$ tiene sentido la recurrencia
	\item Para $n=3$ no tiene sentido la recurrencia
	\item Para $n=4$ tiene sentido la recurrencia
\end{itemize}

\smallbreak Parece que la recurrencia tiene sentido para valores de n que son potencias de 2 \smallbreak

\qquad Por lo que se puede decir que $n = 2^{k}$ \smallbreak

\begin{tabular}{|c|c|c|}
	\hline
	k & n = $2^{k}$ &T(n) \\ \hline
	0 & 1 & T(1) = 3 \\ \hline
	1 & 2 & $T(2) = 2 + 2(3) = 8$ \\ \hline
	2 & 4 & $T(4) = 2 + 2[2+2(3)] = 2 + 2^{2} + 2^{2}*3 = 18$ \\ \hline
	3 & 8 & $T(8) = 2 + 2(2+2[2+2(3)]) = 2 + 2^{2} + 2^{3} + 2^{3}*3 = 38$ \\ \hline
	4 & 16 & $T(16) = 2 + 2[2+2(2+2[2+2(3)])] = 2 + 2^{2} + 2^{3} + 2^{4} + 2^{4}*3 = 78$ \\ \hline
\end{tabular}
\newpage

Hasta ahora en $T(16)$ se tiene algo de la forma:
\begin{equation*}
\begin{aligned}
	T(16) = 2^{1} + 2^{2} + 2^{3} + 2^{4} +2^{4}*3 \\
\end{aligned}
\end{equation*}

Se puede ver el siguiente patrón:
\begin{equation*}
	\begin{aligned}
		T(2^{k}) = 2^{k}*3 + \sum_{i=1}^{k} 2^{i} \\
	\end{aligned}
\end{equation*}

Convertir la sumatoria para que esté en términos de la serie geométrica de la forma $\displaystyle \sum_{i=1}^{n}ar^{i-1}$
\begin{equation*}
	\begin{aligned}
		\sum_{i=1}^{k}2^{i} \ \Rightarrow \ \sum_{i=1}^{k}2^{i} -2^{k} \\
	\end{aligned}
\end{equation*}

Simplificar la sumatoria usando la fórmula de la serie geométrica $S_{n} = \frac{a(r^{n}-1)}{r-1}$, de la sumatoria tenemos $a = 1 \ ; \ r=2 \ ; \ n =k+1$ \smallbreak
Sustituyendo
\begin{equation*}
	\begin{aligned}
		\sum_{i=2}^{k+1}2^{i-1} = \frac{2^{k+1}-1}{2-1} = 2^{k+1}-1
	\end{aligned}
\end{equation*}

\newpage
\subsection{Resolviendo la segunda recurrencia}
\begin{equation*}
	T(n) =
	\begin{cases}
		\begin{aligned}
			&5 \ si \ n = 1\\
			&3 + 4T(n-1) \ si \ n > 1
		\end{aligned}
	\end{cases}
\end{equation*}

\end{document}